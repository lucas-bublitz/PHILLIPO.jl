
% Introdução

\chapter{INTRODUÇÃO}

O Método de Elementos Finitos consite, em suma, dividir um sitema em elementos discretos, ditos também subdomínios, que, quando combinados, conseguem representar seu comportamento original de maneira satisfatória. \cite{Bittencourt} Essa definição pode parecer um tanto vaga, ou até mesmo obscura, pois não revela a

\cite{Bittencourt}

O estudo de elementos finitos remonta a meados do século passado, e tem seu marco-zero quando cunhado por Ray Claugh em sua obra intitulada \textit{The Finite Element in Plane Stress Analysis}, publicada em 1960. Porém, as técnicas que sustentam o método de elementos finitos já eram conhecidas e, mesmo que empregadas de amodo a não serem reconhecidas como hoje, sd,  \cite{Azevedo}