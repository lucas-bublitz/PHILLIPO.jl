
% Introdução

\chapter{INTRODUÇÃO}

O estudo da resistência dos materiais pode ser ter sua trajetória trassada desde a época de Galileu, que estudou o comportamento de vigas e hastes postos a suportar cargas. Entretanto, esse estudo só ganhou forma definida, com emprego de simbologia e métodos específicos, no século XVIII por eruditos que marcaram seus nomes na história, como Saint-Venant, Poisson, Lamé e Navier. Eles estudaram e modelaram problemas fundamentais, porém, com o avanço da humanidade, problemas mais complexos surgiram, que necessitavam de métodos matemáticos e computacionais mais avançados, o que fez com que a Resistência dos Materiais, ou Mecânica dos Sólidos, expandir-se a outras duas áreas do conhecimento: A Teoria da Elasticidade e a Teoria da Plasticidade. \cite{Hibbeler}






O Método de Elementos Finitos consiste, em suma, dividir um sistema em elementos discretos, ditos também subdomínios, que, quando combinados, conseguem representar seu comportamento original de maneira satisfatória. \cite{Bittencourt}

O estudo de elementos finitos remonta a meados do século passado, e tem seu marco-zero quando cunhado por Ray Claugh em sua obra intitulada \textit{The Finite Element in Plane Stress Analysis}, publicada em 1960. Porém, as técnicas que sustentam o método de elementos finitos já eram conhecidas e, mesmo que empregadas de modo a não serem reconhecidas como hoje, sd,  \cite{Azevedo}

A capacidade da humanidade de expressar compreender o mundo deu um grande salto com a 