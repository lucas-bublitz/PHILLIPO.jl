
% Introdução

\chapter{INTRODUÇÃO}

\begin{quote}
    I think of myself as an engineer, not as a visionary or 'big thinker.' I don't have any lofty goals.
    (Linus Trovalds)  
\end{quote}

Por que essa citação? Como o leitor pode perceber na abertura de cada capítulo, como também de alguns subcapítulos, há uma citação que resume, pelo menos em parte, a motivação sobre a qual o texto foi construído. Neste caso, a citação é de Linus Torvalds, o criador do kernel Linux, e que, em 1991, criou o sistema operacional GNU/Linux, que é hoje o sistema operacional mais utilizado no mundo, principalmente em servidores. Mas, no caso desse capítulo, essa citação não têm o intuito de ser sobre sistemas operacionais, mas sim sobre a visão realista que, na perspectiva de Linus (e do próprio Autor), o engenheiro deve buscar: aquele que encontra soluções.



A limitação do ser humano em captar, integralmente, os fenômenos ao seu redor é evidente, a ponto de não conseguir compreender como eles se dão; portanto, analisar um fenômeno, separando-o em pequenas partes (ou elementos) cujo comportamento já é conhecido, e, a partir da justaposição delas, reconstruir o funcionamento do próprio fenômeno, é um modo intuitivo que engenheiros e cientistas procedem em seus estudos, criando modelos matemáticos sobre a natureza \cite{Zin}.


O Método dos Elementos Finitos, ou os métodos numéricos em geral, e a programação andam lado a lado, ambos impulsionados pelo avanço do processamento computacional \cite{Onate}. Então, é de se esperar que uma aplicação desse método seja acompanhada de um projeto de software conciso; a abordagem do tema na graduação de engenharia mecânica, entretanto, é focada, fortemente, na aplicação direta dos algoritmos, sem muita observação de assuntos relacionados a programação funcional ou processamento paralelo, o que é devido, dentre outros aspectos, pela falta de cadeiras abordando o desenvolvimento de aplicações, ou mesmo de programação no geral (com exceção de algumas matérias propedêuticas que ensinam introdução à lógica de programação em C).


\section{Motivação}

Os métodos numéricos são fundamentais para, e não somente, a engenharia, visto que soluções analíticas de problemas complexos são, senão impossíveis em muitos casos, difíceis de se obter \cite{Onate}. Dentre eles, o método de elementos finitos é bem conhecido, e tem destaque em duas áreas muito nobres da engenharia mecânica: a análise estrutural e a transferência de calor.

O tema surgiu quando o autor se encontrou na tarefa de adicionar uma funcionalidade em um software já existente de elementos finitos, e percebeu que, mesmo tendo visto o assunto na graduação, não detinha o conhecimento necessário para compreender o seu funcionamento. Então resolveu por criar seu próprio programa, com o paradigma funcional, aplicando seus conhecimento prévios de projeto de software, desenvolvendo mais o seu entendimento sobre o Método dos Elementos Finitos.


\section{Objetivos}

O objetivo geral deste trabalho foi desenvolver uma aplicação de elementos finitos para a análise de estruturas rígidas sobre carregamentos estáticos em regime elástico linear, utilizando para tanto, um paradigma de programação funcional com alguns aspectos de paralelismo, e focando em características modulares de implementação e de legibilidade do código, com o intuito secundário de demonstrar as facilidades e vantagens da linguagem Julia no âmbito da programação científica.

\subsection{Objetivos específicos}

\begin{enumerate}
    \item Estudar o MEF aplicado na determinação de deformações de estruturas sólidas em regime elástico e linear, sob carregamentos estáticos.
    \item Programar os algoritmos de MEF em Julia.
    \item Desenvolver um módulo que seja distribuível pelo gerenciador de pacotes Pkg.jl, em um repositório público hospedado no GitHub.
    \item Aplicar processamento paralelo em determinadas partes do programa em que as funções nativas não o fazem, a fim de utilizar mais da capacidade de processamento do computador que um código feito sobre o paradigma estruturado.
    \item Estudar as características da linguagem Julia, e como ela pode ser uma boa alternativa para C e FORTRAN em programação científica de alta performance.
    \item Propagar a linguagem Julia no meio acadêmico brasileiro, principalmente nas áreas de ciências naturais.
\end{enumerate}

\section{Organização do documento}

Este documento é organizado em quatro grandes partes, que tratam: (1) sobre o método de elementos finitos aplicado no equilíbrio estático de estruturas; (2) sobre a linguagem de programação Julia; (3) sobre PHILLIPO, o módulo desenvolvido; (4) sobre validação e verificação dos resultados. (*) Como o ChatGPT e o GitHub influenciaram e influenciaram o desenvolvimento deste documento e de PHILLIPO.jl.

