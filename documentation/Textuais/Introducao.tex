
% Introdução

\chapter{INTRODUÇÃO}

\begin{quote}
    I think of myself as an engineer, not as a visionary or 'big thinker.' I don't have any lofty goals.
    (Linus Trovalds)  
\end{quote}

O Método dos Elementos Finitos e a programação andam lado a lado, ambos impulsionados pelo avanço do processamento computacional \cite{Zin}. Portanto, é de se esperar que uma aplicação desse método seja acompanhada de um projeto de software conciso, entretanto, a abordagem desse tema na graduação de engenharia mecânica é focada, fortemente, na aplicação direta dos algoritmos, sem muita observação de assuntos relacionados a programação funcional ou processamento paralelo, o que é devido, dentre outros aspectos, pela falta de cadeiras abordando o desenvolvimento de aplicações, ou mesmo de programação no geral (com exceção de algumas matérias propedêuticas que ensinam introdução à lógica de programação com C).

\section{Motivação}


O início do projeto de PHILLIPO foi motivado pela aplicação do MEF na análise estática de estruturas metálicas quaisquer. Entretanto, com o avanço da construção do código, optou-se por, além de implementar esses algoritmos, desenvolver uma base em Julia que pudesse, ao mesmo tempo que fosse legível, ser modular o suficiente a ponto de 

\section{Objetivos}

O principal objetivo deste trabalho foi desenvolver uma aplicação, escrita em Julia, de elementos finitos para a análise de estruturas rígidas sobre carregamento estático em regime elástico linear, utilizando para tanto, um paradigma de programação funcional com alguns aspectos de paralelismo, e focando em características modulares de implementação e de legibilidade do código.

\subsection{Objetivos específicos}

\begin{enumerate}
    \item Estudar o MEF aplicado na determinação de deformações de estruturas rígidas em regime elástico e linear.
    \item Programar os algoritmos de MEF em Julia.
    \item Desenvolver um módulo que seja distribuído pelo gerenciador de pacotes Pkg.jl, em um repositório público hospedado no GitHub.
    \item Aplicar processamento paralelo em determinados pontos do programa, a fim de utilizar mais da capacidade de processamento do computador que um código feito sobre o paradigma estruturado.
    \item Estudar as características da linguagem Julia, e como ela pode ser uma boa alternativa para C e FORTRAN em programação científica de alta performance.
    \item Propagar a linguagem Julia no meio acadêmico brasileiro, principalmente nas áreas de ciências naturais.
\end{enumerate}

