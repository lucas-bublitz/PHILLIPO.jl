
% Introdução

\chapter{INTRODUÇÃO}

\begin{quote}
    I think of myself as an engineer, not as a visionary or 'big thinker.' I don't have any lofty goals.
    (Linus Trovalds)  
\end{quote}

A limitação do ser humano em captar, integralmente, os fenômenos ao seu redor é evidente, a ponto de não conseguir compreender como eles se dão; analisar um fenômeno, portanto, separando-o em pequenas partes (ou elementos) cujo comportamento já é conhecido (simplificado ou não), e, a partir da justaposição delas, reconstruir o funcionamento do próprio fenômeno, é um modo intuitivo que engenheiros e cientistas procedem em seus estudos, criando modelos matemáticos sobre a natureza \cite[p. 2]{Zin}.



O Método dos Elementos Finitos, ou os métodos numéricos em geral, acompanha o desenvolvimento da programação, ambos impulsionados pelo avanço do processamento computacional \cite{Onate}. Então, é de se esperar que uma aplicação desse método seja acompanhada de um projeto de software conciso; a abordagem do tema na graduação de engenharia mecânica, porém, é focada, fortemente, na aplicação direta dos algoritmos, sem muita dedicação em assuntos relacionados à otimização e à programação funcional ou, até mesmo, processamento paralelo, o que pode ser devido, dentre outros aspectos, a falta de cadeiras abordando o desenvolvimento de aplicações, ou mesmo de programação como um todo. Com exceção de algumas matérias propedêuticas que ensinam introdução à lógica de programação em C, a grade curricular do engenheiro mecânico não contempla o projeto de software e portanto faz com os discente que queiram seguir em uma áreas mais computacionais, como simulação de fluidos e estruturas.

O aprofundamento, visto isso, de questões relacionadas à projeto de software é de grande importância para o engenheiro mecânico, pois, além de ser uma ferramenta de trabalho, a programação é uma forma de expressão, e, como tal, deve ser dominada para que o engenheiro possa se comunicar com outros profissionais da área, como programadores e cientistas da computação, e, assim, desenvolver projetos em conjunto. Além disso, o desenvolvimento de um software de elementos finitos é uma forma de estudar o método, e, portanto, de aprofundar o conhecimento sobre o assunto.


\section{Motivação}

O tema surgiu quando o autor se encontrou na tarefa de adicionar uma funcionalidade em um software já existente de elementos finitos, e percebeu que, mesmo tendo visto o assunto na graduação, não detinha o conhecimento necessário para compreender o seu funcionamento. Então resolveu por criar seu próprio programa, com o paradigma funcional, aplicando seus conhecimento prévios de projeto de software, desenvolvendo mais o seu entendimento sobre o Método dos Elementos Finitos.


\section{Objetivos}

O objetivo geral deste trabalho foi desenvolver uma aplicação de elementos finitos para a análise de estruturas sólidas sobre carregamentos estáticos em regime elástico linear, utilizando para tanto, um paradigma de programação funcional\footnotemark[1] com alguns aspectos de processamento paralelo, e focando em características modulares de implementação e de legibilidade do código, com o intuito secundário de demonstrar as facilidades e vantagens da linguagem Julia no âmbito da programação científica.

\footnotetext[1]{O paradigma funcional, tal como é explicado no capítulo \ref{ch:julia} \nameref{ch:julia}, aqui é considerado em \emph{lato senso}, ou seja, admite alguns efeitos colaterais, entretanto, não foge à boa prática estabelecida pela comunidade de Julia.}

\subsection{Objetivos específicos}

Constituem objetivos específicos deste trabalho:

\begin{enumerate}
    \item estudar o MEF aplicado na determinação de deformações de estruturas sólidas em regime elástico e linear, sob carregamentos estáticos;
    \item programar os algoritmos de MEF em Julia;
    \item desenvolver um módulo que seja distribuível pelo gerenciador de pacotes Pkg.jl, em um repositório público hospedado no GitHub;
    \item aplicar processamento paralelo em determinadas partes do programa em que as funções nativas não o fazem, a fim de utilizar mais da capacidade de processamento do computador que um código feito sobre o paradigma estruturado;
    \item estudar as características da linguagem Julia, e como ela pode ser uma boa alternativa para C e FORTRAN em programação científica de alta performance;
    \item propagar a linguagem Julia no meio acadêmico brasileiro, principalmente no campus CCT da UDESC.
\end{enumerate}

\section{Organização do documento}

Este documento aborda o projeto e o desenvolvimento de um módulo em Julia, denominado PHILLIPO.jl, que aplica o Método de Elementos Finitos, integrado à ferramenta de pré e pós-processamento GiD, para realizar a análise das tensões em estruturas sólidas e elásticas sobre carregamentos constantes; e é organizado em cinco partes, que tratam sobre:

\begin{enumerate}
    \item a mecânica dos sólidos: tensão e deformação no regime elástico
    \item o método de elementos finitos aplicado no equilíbrio estático de estruturas sólidas;
    \item a linguagem de programação Julia: o processamento paralelo acessível a engenheiros;
    \item validação e verificação de resultados;
    \item PHILLIPO.jl, o módulo desenvolvido.
    \item Comentários sobre o Copilot e o ChatGPT no desenvolvimento de PHILLIPO.jl e deste documento.
\end{enumerate}
