
% Introdução

\chapter{INTRODUÇÃO}

\begin{quote}
    I think of myself as an engineer, not as a visionary or 'big thinker.' I don't have any lofty goals.
    (Linus Trovalds)  
\end{quote}


O Método dos Elementos Finitos e a programação andam lado a lado, ambos impulsionados pelo avanço do processamento computacional. Portanto, é de se esperar que uma aplicação do método seja acompanhada de um projeto de software estruturado, entretanto, a abordagem desse tema na graduação de engenharia mecânica é focada, fortemente, na aplicação direta dos algoritmos, sem muita observação de assuntos relacionados a programação funcional ou processamento paralelo, o que é devido, dentre outros aspectos, pela falta de cadeiras abordando o desenvolvimento de aplicações. 

\section{Motivação}

A iniciativa do projeto surgiu quando o autor se deparou com a tarefa de adicionar uma nova funcionalidade em outro programa de elementos finitos. E, ao analisar o código, o autor 

O projeto de PHILLIPO foi motivado, primeiramente, para realizar a aplicação do MEF na análise estática de estruturas metálicas quaisquer. Entretanto, com o avanço da construção do código, optou-se por, além de implementar esses algoritmos, desenvolver uma base em Julia que pudesse, ao mesmo tempo que fosse legível, ser modular o suficiente a ponto de 

\section{Objetivos}

O principal objetivo deste trabalho foi desenvolver uma aplicação, escrita em Julia, de elementos finitos para a análise de estruturas rígidas sobre carregamento estático em regime elástico linear, utilizando para tanto, um paradigma de programação funcional com alguns aspectos de paralelismo, e focando em características modulares de implementação e de legibilidade do código.

\subsection{Objetivos específicos}

\begin{enumerate}
    \item Estudar o MEF aplicado na determinação de deformações de estruturas rígidas em regime elástico e linear.
    \item Estudar as características da linguagem Julia, e como ela pode ser uma boa alternativa para programação científica de alta performance.
\end{enumerate}