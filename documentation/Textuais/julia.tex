\chapter{Julia}

\begin{quotation}
    A programing language to heal the planet together.
    (Alan Edelman)
\end{quotation}

Julia é uma linguagem de programação dinâmica, opcionalmente tipada, pré-compilada, generalista, de código livre (\citeshort{Sherrington}) de alto nível. \cite[The scope of Julia]{Sherrington}




\section{Fortran e C, legados da computação de alta performance}

Os programas de simulações físicas (de MEF, MVF etc.), não comerciais, são comumente escritos em C ou Fortran, com interfaces escritas em Python. Isso se deve pela performance que os compiladores dessas linguagens oferecem, como também pela ampla referência, tanto por documentações já bem consolidadas, quanto por uma certa convenção do mundo acadêmico. Entretanto, manter essa tradição promove algumas dificultados na hora de escrever e depurar o código.


Há sempre uma barganha que o programador tem que fazer na hora de decidir que linguagem será adotada para o desenvolvimento de uma projeto de software: legibilidade ou performance. Alguns autores podem se referir a esse dilema como abstração vs. desempenho ou simplicidade vs. desempenho, embora o termo abstração não seja um sinônimo de lentidão, nem a simplicidade, uma característica inerente à performance. Em suma, ou o código é fácil de ser lido por um ser humano, ou pode ser executado rapidamente, e é fácil de se otimizar. Engenheiros e cientistas, atualmente, começam sua jornada como programadores de linguagens dinâmicas, que são fáceis de escrever e ler, Python. Entretanto, quando avançam para áreas de simulação ou tratamento de dados, fica evidente que essa linguagem não mais tem a performance suficiente para resolver os problemas em tempo hábil,

Python e C são bons exemplos disso, embora não exista em si uma linguagem mais rápida que a outra, pois elas são apenas abstrações de comandos interpretados pelo compilador ou interpretador, é notável que programs escritos puramente em Python tendem a serem executados mais lentamente que programas em C. Isso se devem, entre outras causas, por C ter uma integração maior com o hardware, possibilitando o programador a remodelar o algoritmo de modo a tornar mais eficiente sua execução, ou seja, a interação entre o programador e a máquina é muito próxima direta, pois ele programa já com a estrutura do computador em mente. Python, por sua vez, cria uma camada a mais. Os comandos escritos nessa linguagem são interpretados dinamicamente gerando um bytecode, para então serem executados em uma máquina virtual a fim de serem executados. Esse tipo de abordagem tem diversas vantagens, como tornar o código mais legível (esse conceito está vinculado, diretamente, com a semelhança do código com o algoritmo que o programador tem por objetivo implementar) e, de certa forma, universal, pois com essa nova camada permite que qualquer código em Python possa rodar em qualquer computador que tenha o interpretador da linguagem instalado, salvo casos de dependências externas. Entretanto, essa camada extra faz com que o interpretador não possa mais encontrar otimizações com facilidade, distanciando o código do que é realmente executado no hardware. Julia foi a solução desenvolvida para conciliar esses dois mundos.

\section{O que Julia pode substituir (sem muita dificuldades)}

Mesmo com o avanço notável que a linguagem tem tido na última década, ainda não é óbvio que se possa, sem pensar duas vezes, suprir a demanda de programas robustos e otimizados para o processamento e gerenciamento de quantidades massivas de dados. Não é óbvio justamente pelo fato de não ser verdade. Em lato senso, Julia é C. Em stricto senso, nem perto disso. Para poder explicar esse tópico um tanto polêmico, sobre a eficiência das linguagem interpretadas, compiladas ou pré-compiladas (categorias que hoje não refletem mais, com clareza, a forma de execução das linguagens de programação), é preciso dar um passo atrás e começar com uma pergunta mais fundamental: o que é uma programa? Ainda mais que o presente trabalho trata da aplicação de Julia em uma área que, comumente, não se preocupa com esse tópico.

