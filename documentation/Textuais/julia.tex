\chapter{Julia \& seus módulos} 
\label{ch:julia}

\begin{quotation}
    A programming language to heal the planet together.
    (Alan Edelman)
\end{quotation}

\begin{figure}[hbtp]
    \centering
    \caption{Logo da linguagem Julia}
    \includegraphics[scale = 0.6]{Figuras/julia-logo-color.pdf}
    \label{fig:julia-logo}
\end{figure}

Julia é uma linguagem de programação dinâmica, opcionalmente tipada, pré-compilada, generalista, de código livre\footnote{A Linguagem Julia, é distribuída, quase integralmente, sob a MIT License, que permite a modificação, utilização e distribuição, seja comercial ou não, de qualquer parte do código, assim como das documentações associadas. Os componentes do módulo Base e as bibliotecas e ferramentas externas, que têm licença diferente, assim como a da própria Julia, podem ser consultados diretamente no repositório da linguagem: \url{https://github.com/JuliaLang/julia}.} e de alto nível, criada por Jeff Bezanso, Stefan Karpinski, Viral B. Shah e Alan Edelman, em 2012, com o objetivo de minimizar o problema das duas linguagens (\emph{the two language problem}). É voltada para a programação científica, com capacidades de alta performance e sintaxe simples, similar à notação matemática usual. \cite{Sherrington}

Na aplicação do MEF, a linguagem se destaca por sua sintaxe próxima à linguagem matemática, principalmente na manipulação de matrizes e vetores, que permite a implementação de algoritmos de forma simples e rápida, e por sua capacidade de processamento paralelo, que permite a otimização na construção e resolução de sistemas algébricos grandes. Além disso, o empacotamento oferecido pela linguagem, por meio do \emph{Pkg.jl}, fornece ferramentas de controle e versionamento, como também, a criação de pacotes, que são facilmente compartilhas e instalados.

Este capítulo aborda os seguintes tópicos  que são relevantes para a aplicação do MEF em Julia, e que foram explorados no desenvolvimento de PHILLIPO:

\begin{enumerate}
    \item sintaxe;
    \item empacotamento com \emph{Pkg.jl};
    \item processamento paralelo com \emph{Threads.jl};
    \item matrizes esparsas com \emph{SparseArrays.jl};
    \item performance de Julia;
\end{enumerate}

\section{O escopo de Julia}

\begin{quotation}
    "In short, because we are greedy."
    (Jeff Bezanson, Stefan Karpinski, Viral B. Shah e Alan Edelman, em \emph{Why We Created Julia})
\end{quotation}

A linguagem Julia é rápida. Tão rápida \cite{julia}