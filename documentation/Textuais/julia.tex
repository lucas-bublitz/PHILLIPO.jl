\chapter{Julia}

\begin{quotation}
    
\end{quotation}

Julia é rápida. Não como C ou Fortran, mas rápida o suficiente para que barganhas de performance-praticidade sejam apenas notáveis em programas mal otimizados, ou com parâmetros de execução mal definidos. Em suma, a motivação por trás de escolher Julia é a praticidade, afinal, qual é o tempo mais preciso: o da máquina ou o do programador?

Mesmo com o avanço notável que a linguagem tem tido na última década, ainda não é óbvio que se possa, sem pensar duas vezes, suprir a demanda de programas robustos e otimizados para o processamento e gerenciamento de quantidades massivas de dados. Não é óbvio justamente pelo fato de não ser verdade. Em lato senso, Julia é C. Em stricto senso, nem perto disso. Para poder explicar esse tópico um tanto polêmico, sobre a eficiência das linguagem interpretadas, compiladas ou pré-compilas (categorias que hoje não refletem mais, com clareza, a forma de execução das linguagens de programação), é preciso dar um passo atrás e começar com uma pergunta mais fundamental: o que é uma programa?
K