\chapter{Julia \& seus módulos} 
\label{ch:julia}

\begin{quotation}
    A programming language to heal the planet together.
    (Alan Edelman)
\end{quotation}

\begin{figure}[hbtp]
    \centering
    \caption{Logo da linguagem Julia}
    \includegraphics[scale = 0.6]{Figuras/julia-logo-color.pdf}
    \label{fig:julia-logo}
\end{figure}

Julia é uma linguagem de programação dinâmica, opcionalmente tipada, pré-compilada, generalista, de código livre\footnote{A Linguagem Julia é distribuída, quase integralmente, sob a MIT License, que permite a modificação, utilização e distribuição, seja comercial ou não, de qualquer parte do código, assim como das documentações associadas.} e de alto nível, criada por Jeff Bezanso, Stefan Karpinski, Viral B. Shah e Alan Edelman, em 2012, com o objetivo de minimizar o problema das duas linguagens (\emph{the two language problem}). É voltada para a programação científica, com capacidades de alta performance e sintaxe simples, similar à notação matemática usual. \cite{Sherrington}

Na aplicação do MEF, a linguagem se destaca por sua sintaxe semelhante a do MATLAB, que se mostra ideal na manipulação de matrizes e vetores, e por sua capacidade de processamento paralelo, que permite a otimização na construção e resolução de sistemas algébricos grandes. Além disso, o empacotamento oferecido pela linguagem, por meio do \emph{Pkg.jl}, fornece ferramentas de controle e versionamento, como também, a criação de módulos, uma forma eficiente de organizar e distribuir aplicações dentro o ecossistema de Julia.

Este capítulo aborda os seguintes tópicos que são relevantes para a aplicação do MEF em Julia, e que foram explorados no desenvolvimento de PHILLIPO:

\begin{enumerate}
    \item matrizes esparsas com \emph{SparseArrays.jl};
    \item empacotamento com \emph{Pkg.jl};
    \item processamento paralelo com \emph{Threads.jl};
\end{enumerate}



\section{Matrizes esparsa e \emph{SparseArrays.jl}}

As matrizes esparsas desempenham um papel crucial em diversas áreas da computação científica, sendo particularmente relevantes no contexto do MEF. Entender o conceito de matrizes esparsas é fundamental para otimizar a resolução de sistemas lineares, economizando recursos computacionais e acelerando significativamente o tempo de resolução de sistemas lineares grandes.

Uma matriz esparsa é aquela em que a maioria dos elementos é igual a zero (em contraste com as matrizes densas, onde a maioria dos elementos é diferente de zero). Na aplicação do MEF, a matriz de rigidez global tem essa characteristics, quando o problema é composto por um número grande de elementos, que pode ser explorada tanto na forma de armazena-lá quanto na solução de sistemas lineares \cite{LOGAN}. Aqui são apresentadas duas formas de armazenamento de matrizes esparsas, e como elas podem ser utilizadas em Julia pelo módulo \emph{SparseArrays.jl}: COO e CSR.


O armazenamento de matrizes esparsas por coordenadas (COO) consiste em gravar apenas os valores não nulos em um conjunto de tuplas $(i, j, v)$, em que $i$ e $j$ são os índices de linha e coluna, respectivamente, e $v$ é o valor na posição correspondente. Todas outras posições são presumidas nulas. Uma matriz esparsa, por exemplo
\begin{equation}
    A = 
    \begin{bmatrix}
        1 & 0 & 0 & 1 \\
        0 & 2 & 0 & 0 \\
        0 & 0 & 3 & 1 \\
        0 & 0 & 0 & 4 \\
    \end{bmatrix},
    \label{eq:coo_matrix}
\end{equation}
pode ser armazenada no conjunto de tuplas
\begin{equation}
    A_{COO} = {(1,1,1), (2,2,2), (3,3,3), (4,4,4), (1,3,1), (3,4,1)}.
    \label{eq:coo}
\end{equation}

Putra 
