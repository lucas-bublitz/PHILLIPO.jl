\chapter{Objetivos alcançados e melhorias em projetos futuros}

Todos os objetivos específicos, inicialmente, foram alcançados. O MEF foi devidamente compreendido e aplicado na análise de estruturas sólidas elásticas sob carregamentos contantes, por meio dos elemento triangulares e tetraédricos lineares. Os algoritmos do MEF foram implementados com sucesso na linguagem Julia, em um programa, nomeado PHILLIPO.jl, distribuível, e empacotado pelo gerenciador \emph{Pkg.jl}. Nesse programa, foi aplicado o processamento paralelo na montagem da matriz de rigidez global.

O programa, entretanto, tem grande margem para melhorias, tanto em termos de desempenho e funcionalidades, quanto em legibilidade. A seguir, são listadas algumas melhorias para projetos futuros no desenvolvimento de PHILLIPO.jl:

\begin{enumerate}
    \item \textbf{Estudar a estrutura de implementações já consolidadas do MEF.} Estudar outras implementações do MEF, especialmente em Julia, é um modo produtivo de se aprimorar o programa, observando os pontos fortes e fracos de suas estruturas. Módulos como o \emph{JuliaFEM.jl} vem sendo desenvolvidos há anos, já possuem uma comunidade ativa considerável.
    \item \textbf{Aprimorar o empacotamento e a estrutura do programa.} O arquivo principal de PHILIPO.jl ainda se encontra com várias etapas do MEF aplicadas explicitamente, o que prejudica a leitura do código. Cálculos de graus de liberdade, estruturas condicionais para distinguir entre casos bidimensionais e tridimensionais são alguns exemplos de procedimentos que poderiam estar em funções separadas, para serem apenas chamadas na função principal, o que facilitaria a leitura e a manutenção do código. A estrutura do programa também pode ser revista, com o intuito de facilitar a implementação de novos elementos e, até, novas equações de governo e novos tipos de análise.
    \item \textbf{Implementar melhores formas de debug e mensagens de erro.} O debug e as mensagens de erro foram implementados utilizando impressões no terminal, concatenadas no arquivo \emph{.log}, o que é não é um prática recomendável em Julia. Na própria documentação da linguagem, é indicado um conjunto de macros para essas finalidades: \emph{@warn} e \emph{@debug}, justamente para para facilitar a leitura e manutenção do código.
    \item \textbf{Aplicar mais o paradigma de despachos múltiplos.} Embora esse é um ponto notável na linguagem Julia, o programa ainda não faz uso de todo o potencial dessa característica. A função de montagem da matriz global de rigidez poderia ser implementada sem a utilização de estruturas condicionais, o que tornaria o código mais legível e reaproveitável, ao passo que a inclusão de outros tipos de elementos não necessitaria de alterações nessa função.
    \item \textbf{Implementar integração com outras interfaces.} O GID, embora muito poderoso para gerar malhas e plotar resultados, é limitado pela sua licença (não é um projeto open-source), e pela sua comunidade que não é muito ativa. O GMSH e o Salome se mostram como alternativas viáveis e gratuitas.
    \item \textbf{Realizar testes de desempenho.} Embora o programa tenha aplicado matrizes esparsas e processamento paralelo, não foi realizado nenhum teste formal de desempenho, comparando resultados com outros programas similares.
    \item \textbf{Implementar de outros tipos de elementos.} Este projeto se ateve a implementar elementos simples, mas, com algumas alterações, é possível adicionais elementos mais complexos em PHILLIPO.jl, o que implicaria, dentre outras coisas, na criação de um novos módulos internos para abarcar novas funcionalidade, como transformação de sistemas de referência e integração numérica.
    \item \textbf{Implementar de outros tipos de análise.} Não somente análises de deformação e tensão podem ser feitas com o MEF, mas virtualmente de qualquer fenômeno físico que possa ser descrito por equações diferenciais, como condução de calor e vibrações.
    \item \textbf{Criar uma documentação didática e completa.} Como o objetivo deste trabalho foi expor o MEF em Julia, não foi abordado minuciosamente todos os aspectos do programa, como poderia ter sido realizado em um projeto de maior escopo. Uma proposta futura interessante é a criação de um documento mais completo que, além de explicar o funcionamento detalhado do programa, traria tópicos da própria linguagem Julia.
\end{enumerate}

