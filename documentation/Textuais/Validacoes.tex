

%334% 
$$ asdasd $$
\chapter{Validação \& Verificação}


O desenvolvimento de softwares de simulação, seja utilizando o MEF ou não, é sempre acompanhado de uma bateria de testes, além de um procedimento de validação e verificação (V \& V), que garante, dentro de uma margem de abrangência do que se propõe, sua capacidade de reproduzir resultados concisos, sendo, então, um espelho da realidade.

Verificação, dentro desse contexto, é o procedimento pelo qual se evidencia a exata implementação do modelo matemático no próprio software, ou seja, a verificação que a modelagem programada é equivalente ao algoritmo matemático, dentro dos limites impostos pela aritmética computacional em relação às operações em ponto flutuante\footnote{No computador, os números ditos Reais ($\mathbb{R}$) são representandos por um ponto flutuante, que é uma forma discreta, pois os computadores são desenvolvidos em lógica booleana}. Já a validação, por sua vez, é o processo para determinar a acurácia que um modelo computacional possui de presentar a realidade, dentro dos limites que se propõe. Essas definições estão de acordo com o documento \emph{An Overview of the Guide for Verification and Validation
in Computational Solid Mechanics}, referente a norma ASME respectiva. Por questões de simplificação, a verificação foi realizada por comparação com outro programa semelhante, o Abaqus, e a validação, por comparação com resultados analíticos, já consolidados experimentalmente.

\section{Verificação}

Foram utilizados 


s