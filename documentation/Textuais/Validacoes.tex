\chapter{Validação \& Verificação}

Para que um programa seja considerado válido, é preciso que passe tanto por uma bateria de testes, quanto por um processo de verificação \& validação (V\&V). O que ocorre, também, em programas sobre MEF.
Verificação é o processo pelo qual se determina se um modelo computacional tem acurácia suficiente para representar o modelo matemático em que é embasado. Validação, por sua vez, é o processo para determinar a acurácia que um modelo computacional possui de representar a realidade, dentro dos limites que se propõe. (ASME)

