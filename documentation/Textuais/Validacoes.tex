

%334% 
$$ asdasd $$
\chapter{Validação \& Verificação}


O desenvolvimento de softwares de simulação, seja utilizando o MEF ou não, é sempre acompanhado de uma bateria de testes, além de um procedimento de validação e verificação, que garante, dentro de uma margem de abrangência do que se propõe o software, sua capacidade de reproduzir resultados concisos, sendo, então, um espelho da realidade. 

Verificação, dentro desse contexto, é o procedimento pelo qual se evidencia a exata implementação do modelo matemático no próprio software, ou seja, a verificação que a modelagem programada é equivalente ao algoritmo matemático, dentro dos limites impostos pela aritmética computacional em relação às operações em ponto flutuante\footnote{No computador, os números ditos Reais ($\mathbb{R}$) são representandos por um ponto flutuante, que é uma forma discreta, pois os computadores são desenvolvidos em lógica booleana}.



Para que um programa seja considerado válido, é preciso que passe tanto por uma bateria de testes, quanto por um processo de verificação \& validação (V\&V). O que ocorre, também, em programas sobre MEF.
Verificação é o processo pelo qual se determina se um modelo computacional tem acurácia suficiente para representar o modelo matemático em que é embasado. Validação, por sua vez, é o processo para determinar a acurácia que um modelo computacional possui de representar a realidade, dentro dos limites que se propõe. (ASME)

