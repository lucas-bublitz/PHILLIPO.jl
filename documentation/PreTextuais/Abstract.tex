
% Abstract

\begin{resumo}[Abstract]
 \begin{otherlanguage*}{english}
  The Finite Element Method (FEM) is a robust mathematical tool for analyzing physical phenomenon described by differential equations, such as stress and strain in solids. It involves discretizing the domain of the unknown field into elements, where the physical phenomenon is approximated using interpolation functions. This process leads to systems of algebraic equations, and their solution, when assembled, approximates the solution of the original differential equation. Given the computational intensity involved in formulating and solving these systems, there arises a necessity for their implementation in a computational environment. This work focuses on the implementation of the Finite Element Method for analyzing stress and strain in solid bodies under static loads in a linear elastic regime. A module is developed in the Julia programming language, incorporating aspects of parallel processing and packaging as a demonstration. The developed module, named PHILLIPO.jl, is accessible via the Pkg.jl package manager in a public repository on GitHub. Parallelism is applied during the assembly of the global stiffness matrix, and integration with the pre and post-processing interface GID is achieved. Two types of elements are implemented: the Constant Strain Triangle (CST) and the linear tetrahedron. Additionally, four types of boundary conditions are considered: fixed supports, prescribed displacements, nodal concentrated forces, and line and surface loads. The program undergoes Verification \& Validation (V \& V) by comparing displacement and stress results with Abaqus software and the Euler-Bernoulli beam model, demonstrating the correct implementation of the FEM mathematical algorithm. Finally, potential areas for improvement in future projects are identified, emphasizing code enhancement in terms of readability and packaging, as well as the implementation of new features and performance testing.
  
  \textbf{Keywords}: Finite Element Method (FEM), Julia (Computer Programming Language), Structural Analysis, Mechanical Engineering.
 \end{otherlanguage*}
\end{resumo}
