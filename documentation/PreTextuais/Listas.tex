
% ---
% inserir lista de ilustrações
% ---
\pdfbookmark[0]{\listfigurename}{lof}
\listoffigures*
\cleardoublepage
% ---

% ---
% inserir lista de quadros
% ---
%\pdfbookmark[0]{\listofquadrosname}{loq}
%\listofquadros*
%\cleardoublepage
% ---


% ---
% inserir lista de tabelas
% ---
\pdfbookmark[0]{\listtablename}{lot}
\listoftables*
\cleardoublepage
% ---

% ---
% inserir lista de abreviaturas e siglas
% ---
\begin{siglas}
	\item[MEF] Método dos Elementos Finitos
	\item[FEM] Finite Element Method 
	\item[SI]  Sistema Internacional de Unidades
	\item[COO] COOdinate list
	\item[CSC] Compressed Sparse Column
	\item[CST] Constant Strain Triangle    
\end{siglas}
% ---

% ---
% inserir lista de símbolos
% ---


\begin{simbolos}

  \item[$\bm{\sigma}$] tensor de tensões
  \item[$\bm{\epsilon}$] tensor de deformações
  \item[$\mathcal{B}$] um corpo sólido, elástico, homogêneo e isotrópico, em equilíbrio
  \item[$\sigma_{ij}$] tensão na direção $i$ e sentido $j$
  \item[$\epsilon_{ij}$] deformação na direção $i$ e sentido $j$
  \item[$\Omega$] domínio da função deslocamento que compreende os pontos de um sólido ou elemento
  \item[$\partial\Omega$] fronteira do domínio $\Omega$
  \item[$\bm{\hat{n}}$] um versor
  \item[$\bm{\varphi}$] função de deslocamento
  \item[$\bm{x}$] vetor posição
  \item[$u,v,w$] componentes da função de deslocamento, nas direções $x,y,z$, respectivamente.
  \item[$\bm{C}$] matriz constitutiva
  \item[E] módulo de elasticidade
  \item[$\nu$] coeficiente de Poisson       
  
\end{simbolos}

% ---
