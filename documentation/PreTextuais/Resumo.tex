% ---
% RESUMOS
% ---

% resumo em português
\setlength{\absparsep}{18pt} % ajusta o espaçamento dos parágrafos do resumo
\begin{resumo}
    O Método dos Elementos Finitos (MEF) é uma ferramenta matemática poderosa para analisar fenômenos modelados por equações diferenciais, como a tensão e deformação em sólidos, e consiste na divisão do domínio do campo incógnito em elementos, nos quais a descrição do fenômeno físico é simplificada por meio de funções de interpolação, gerando sistemas de equações algébricas, de forma que, quando justapostos, sua solução aproxima a própria solução da equação diferencial. Esse procedimento requer que sejam feitos muitos cálculo para construir e resolver o sistema, o que leva a necessidade de implementar o método num ambiente computacional. Este trabalho visa a implementação computacional do Método dos Elementos Finitos, na análise de tensão e deformação em corpos sólidos sob carregamentos estáticos em regime elástico linear, por meio do desenvolvimento de um módulo escrito na linguagem de programação Julia, utilizando aspectos de processamento paralelo, e algumas características de empacotamento, com o objetivo de expor o método e servir de exemplo menor. Para tanto, foi desenvolvido um módulo denominado PHILLIPO.jl, disponível pelo gerenciador de pacotes Pkg.jl num repositório público hospedado no GitHub, com paralelismo aplicado na montagem da matriz global de rigidez, e integrado na interface de pré e pós-processamento GID. Foram implementados dois tipos de elementos: o triângulo de deformações constantes (CST) e o tetraedro linear, e quatro tipo de condições de contorno: engastes, deslocamentos prescritos, forças concentradas nodais e carregamentos em linhas e superfícies. O programa passou por uma etapa de Verificação \& Validação (V \& V), comparando resultados de deslocamento e tensão com o software Abaqus e o modelo de viga de Euler-Bernoulli, demonstrando que o algoritmo matemático do MEF foi implementado corretamente. Por fim, foram elencados ponto de melhoria para projetos futuro, que frisam o aprimoramento do código, em quesitos de legibilidade e empacotamento, e a implementação de novas funcionalidades, como também a realização de testes de desempenho.


    \textbf{Palavras-chave}: Método dos Elementos Finitos (MEF). Julia (Linguagem de Programação de Computador). Análise estrutural. Engenharia Mecânica.

\end{resumo}
